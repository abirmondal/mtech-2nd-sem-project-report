\chapter{Introduction}

Artificial intelligence (AI) is transforming various creative tasks, including story writing \cite{chakrabarty_art_2024, fang_systematic_2023}. This area, known as Automatic Story Generation (ASG), involves creating narratives from initial prompts or data \cite{chhun_language_2024}. The ability to generate stories is seen as a significant demonstration of human creativity, requiring both generating interesting ideas and expressing them skillfully \cite{chhun_language_2024}. Strong ASG systems hold potential for diverse applications such as gaming, education, mental health support, and marketing \cite{chhun_language_2024}.

Historically, computational narrative systems initially relied on approaches like graph-based planning formalisms and custom rules to structure narratives \cite{akoury_storium_2020, alhussain_automatic_2022}. Early statistical models also emerged, focusing on script learning to predict events within a sequence \cite{alhussain_automatic_2022}. More recently, the field has seen significant advances by adopting deep learning models \cite{akoury_storium_2020, alhussain_automatic_2022}, particularly recurrent neural networks (RNNs) and transformer models like Generative Pre-trained Transformers (GPTs) and BART \cite{alabdulkarim_goal-directed_2021, jadhav_research_2024, khan_storygenai_2023}. These neural language models learn to generate story continuations by sampling from learned probability distributions based on large text corpora \cite{alabdulkarim_goal-directed_2021, jadhav_research_2024}. Specialised models have been developed, such as those using event representations to translate semantic information into events (event2event) and then back into human-readable sentences (event2sentence) \cite{khan_storygenai_2023, martin_event_2018}. Other techniques include hierarchical models that first generate a story premise and then expand it into a full text \cite{alhussain_automatic_2022}.

Despite these advances, ASG presents significant challenges. Generating coherent stories, especially over longer ranges, remains difficult \cite{khan_storygenai_2023}. AI-generated narratives can sometimes suffer from issues like repetitive phrases, conflicting logic, or a lack of compelling plot twists. Ensuring creativity in AI-generated content is another complex issue, as large language models can sometimes regurgitate text seen during pre-training \cite{fang_systematic_2023, chakrabarty_art_2024}.

Recognizing the cognitive load often experienced by budding authors \cite{akoury_storium_2020}, researchers have explored machine-in-the-loop storytelling, where AI assists the author by generating sentences or paragraphs \cite{akoury_storium_2020}. This concept of human-AI collaboration is seen as a promising avenue, allowing AI to serve as a muse, scribe, or even critic to empower human creativity and explore new forms of storytelling \cite{fang_systematic_2023, jadhav_research_2024, saddhono_new_2024, }. Studies suggest that human-AI collaboration can effectively improve story creation \cite{akoury_storium_2020}. Platforms like STORIUM facilitate this collaborative storytelling. Research also indicates that AI assistants should be designed to respect human authors' personal values and writing strategies \cite{fang_systematic_2023}.

Evaluating the quality of generated stories is crucial but challenging \cite{alhussain_automatic_2022, chhun_language_2024}. While automatic metrics such as BLEU, ROUGE, and BERTScore are used, they have been shown to correlate moderately to poorly with human judgment, especially for open-ended tasks like story generation \cite{khan_storygenai_2023, chhun_language_2024, kim_multi-modal_2023}. Consequently, human evaluation remains a vital component of assessing story quality and creativity \cite{fan_hierarchical_2018, bradley_quality-diversity_2023, chakrabarty_art_2024, chhun_language_2024}. Researchers have also worked on developing systematic rubrics and tests, such as the Torrance Tests for Creative Writing (TTCW), adapted by experts to evaluate aspects of creative text \cite{chakrabarty_art_2024}.

The research in this field relies on diverse datasets, including large collections of movie plots \cite{khan_storygenai_2023}, short stories like the ROCStories corpus \cite{tang_textbox_2022}, writing prompts \cite{khan_storygenai_2023, fan_hierarchical_2018}, and data from collaborative platforms like STORIUM \cite{akoury_storium_2020}. To support ongoing research and development, open-source libraries like TextBox have been developed, providing unified frameworks for implementing and evaluating various text generation models across different tasks, including story generation \cite{tang_textbox_2022, li_textbox_2021}.

The continuous evolution of AI, particularly large language models, presents exciting possibilities for advancing the capabilities of automatic story generation and enhancing human creativity through collaborative tools and frameworks \cite{suh_luminate_2024}.