\chapter{Prerequisites}

Before proceeding, these are the prerequisites:

\begin{itemize}
    \item \textbf{Events / Event Representations / Event Sequences}: Structured or semi-structured representations of actions or happenings in a story, often extracted from text, used as building blocks for generating narrative content.
    \par
    Example:
    \begin{itemize}
        \item \textbf{Sentence 01}: Mary emailed Jim and he responded to her immediately.
        \par
        \textbf{Events}: \texttt{email(mary,jim,·)}, \texttt{respond(jim,mary,·)}
    
        \item \textbf{Sentence 02}: John and Mary went to the store.
        \par
        \textbf{Events}: \texttt{⟨john,go,store,$\phi$⟩}, \texttt{⟨mary,go,store,$\phi$⟩}.
    \end{itemize}
    
    \item \textbf{Syntactic and Coreference Information}: Data derived from analysing the grammatical structure of sentences (syntactic information, dependencies) and identifying which different words or phrases refer to the same real-world entity (coreference information).
    \par
    Example:
    \begin{itemize}
        \item \textbf{Sentence 01}: Mary emailed Jim and he responded to her immediately.
        \par
        \textbf{Coreference Information}: He refers to Jim, Her Refers to Mary.
    
        \item \textbf{Sentence 02}: John and Mary went to the store.
        \par
        \textbf{Syntactic Information}: “went” - verb, “store” - object, “and” - splits into 2 events.
    \end{itemize}
    
    \item \textbf{Prompts / Premises / Titles / Story Context / Input Sentences / Specifications}: Initial text or explicit parameters provided to a system to define the topic, starting point, constraints, or desired characteristics of the generated story.
    \par
    Example:
    \begin{itemize}
        \item \textbf{Prompt}: A person with a high school education gets sent back into the 1600’s and tries to explain science and technology to the people.

        \item \textbf{Title}: The Bike Accident

        \item \textbf{Input Sentences}
        \begin{itemize}
            \item Carrie had just learned how to ride a bike.
            \item She didn't have a bike of her own.
            \item Carrie would sneak rides on her sister's bike.
            \item She got nervous on a hill and crashed into a wall.
            \item The bike frame bent and Carrie got a deep gash on her leg.
        \end{itemize}
    \end{itemize}
    
    \item \textbf{Outlines / Plots / Storylines}: A structured, higher-level representation of the key events, goals, or sequence of happenings that form the backbone of a story.
    \par
    Example:
    \begin{itemize}
        \item \textbf{Prompt}: When the sunny lifeguard, Sunny, confronts a supernatural mystery in her town, she must overcome her clumsiness and a rival lifeguard.
        \par
        \begin{verbatim}
        [
            introduce_character:sunny,
            add_obstacle:supernatural,
            introduce_character:mysterious, add_twist,
            introduce_character:clumsy,
                                introduce_rival_character,
                                level_up_obstacle
        ]
        \end{verbatim}
    \end{itemize}
\end{itemize}

\section{Problem Statement}

The challenge of developing an automatic story generation framework that enhances narrative coherence, structural organisation, and long-form storytelling is addressed through a multi-stage pipeline incorporating structured event extraction, graph-based event modelling, and sequence generation.