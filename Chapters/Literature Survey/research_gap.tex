\section{Research Gap}

Research into automated story generation and the integration of artificial intelligence within creative writing processes has revealed several critical challenges and research gaps that warrant further investigation. While significant advancements have been made, particularly with the advent of large language models, several key limitations persist across the field.

\begin{itemize}
    \item \textbf{Challenges in Evaluation}: There is no standardized way to assess the creativity and quality of generated stories. Traditional metrics like BLEU and ROUGE are inadequate, and LLM-based evaluations face issues like biases and reward hacking.

    \item \textbf{Generating Coherent and High-Quality Narratives}: AI-generated stories often struggle with maintaining long-range coherence, logical consistency, and originality. They frequently default to clichés and repetition, lacking deeper elements like subtext and character motivation.

    \item \textbf{Limitations in Data and Knowledge Representation}: Training datasets may be too artificial or unconstrained, leading to narratives that lack dramatic structure. Knowledge models are costly to develop and often lack common domain knowledge.

    \item \textbf{Challenges in Control and Exploration of Solution Spaces}: AI struggles to control specific story elements, avoiding hallucinations and repetitiveness. Exploring truly novel solutions and optimising diversity axes in generation remains a challenge.

    \item {Gaps in Educational and Practical Applications}: AI’s role in supporting creativity in education is underexplored, especially for secondary students. Existing studies lack large samples and practical insights into AI-assisted writing.

    \item \textbf{Framework Development and Reproducibility}: - Current open-source story generation tools lack modularity and standardisation, making reproducibility and model comparison difficult. Closed-source models further hinder replication.

    \item \textbf{Ethical Considerations}: AI-generated content raises issues like misinformation, bias, and ethical dilemmas surrounding authorship and copyright.
\end{itemize}